%%%%%%%%%%%%%%%%%%%%%%%%%%%%%%%%%%%%%%%%%%%%%%%%%%%%%%%%%%%%%%%%%%%%%%%%%%%%%%%%%%%%%%%%%%%%%
%%									Chapitre 2												%
%%%%%%%%%%%%%%%%%%%%%%%%%%%%%%%%%%%%%%%%%%%%%%%%%%%%%%%%%%%%%%%%%%%%%%%%%%%%%%%%%%%%%%%%%%%%%

\chapter{Les différents métiers au laboratoire SOLEIL}
	\minitoc
	

%%%%%%%%%%%%%%%%%%%%%%%%%%%%%%%%%%%%%%%%%%%%%%%%%%%%%%%%%%%%%%%%%%%%%%%%%%%%%%%%%%%%%%%%%%%%%



% Début du chapitre
			
	

		Pendant mon stage à SOLEIL, j'ai observé les différents métiers qui permettent au synchrotron de fonctionner et aux chercheurs de faire leurs expériences.			

		\section{Ingénieur calcul}
			L'ingénieur calcul vérifie la rigidité des pièces car sous l'effet de variations de la température les métaux se dilatent ou se contractent. Il doit concevoir des pièces pouvant fonctionner dans différentes conditions. Il doit pouvoir estimer les températures pour savoir comment construire les éléments. Pour pouvoir estimer la rigidité des objets, il utilise des méthodes de calculs numériques.
			
			L'ingénieur est spécialiste en calculs scientifiques. Il a fait Bac+5 en école d'ingénieur. Son salaire est de 2250 euros par mois en tant que débutant et aujourd'hui il gagne 4000 euros par mois. Si il veur évoluer dans son métier il peut se spécaliser dans certains domaines.
			
			L'avantage de ce métier est qu'il n'y a pas d'horaires décalés car pas de contraintes d'exploitation.  
		
		\section{Coordinateurs expérience}
			Le coordinateur expérience est responsable du bon fonctionnement de chaque ligne de lumière. Il doit pouvoir intervenir dans tous les domaines de la physique. Il doit aussi connaître les gestes de premier secours et savoir utiliser les machines.
			
			Le coordinateur expérience est ausi là pour surveiller la machine très coûteuse pendant les week-end et la nuit quand les autres personnes ne travaillent pas.
			
			Au synchrotron SOLEIL il y a six coordinateurs expériences. Ils se partagent le travail en trois fois huit heures par journée. Cela veut dire que chaque jour ,il y a trois coordinateurs expérience qui travaillent; un travaille le matin, le deuxième travaille l'apès-midi et en fin de soirée et le troisième travaille pendant la nuit. Ils échangent les postes tous les jours (week-end et jours fériés compris). Pour que chacun puisse avoir des vacances, ils sont six.
			
			Les formations compris pour ce poste sont scientifiques: il faut un bac scientifique et avoir fait des études supplémentaires en physiques. Il faut parler anglais et connaître les termes spécifiques en anglais parce qu'au synchrotron SOLEIL, des chercheurs et scientifiques de tous les pays viennent travailler.
			
			Le salaire est d'environ 3500 euros net par mois.

			Le désavantage de ce métier est qu'il faut dormir le jour quand on travaille la nuit. 
		
		\section{Mécaniciens}
			Lors de la construction de l'accélérateur d'électrons, les mécaniciens ont installé les pièces et construit la machine. 
			
			Maintenant, ils s'occupent de la maintenance de la machine et construisent des pièces avec différentes outils. Ils utilisent la fraiseuse et la tourneuse pour travailler des métaux et créer différentes pièces. Ils utilisent aussi une rectifieuse pour lisser les faces, une plieuse pour plier les différents métaux. 

			Les mécaniciens soudent aussi.
			
			Le revenu est entre 1600 et 2000 euros net mensuels. La formation est un apprentissage en école spécialisée, un CAP, un BEP ou un Bac Pro.
		
		\section{Services achats}
			Au seins du service administratif, certaines personnes s'occupent des achats. Pour cela il y a une assistante, trois acheteurs et deux personnes qui passent les commandes. Les catégories d'achats sont les fournitures, les services, et les travaux.
		
		\section{Videurs}
			Certaines personnes s'occcupent de vérifier si les tubes sont bien sous vide. en effet les tubes ne doivent pas contenir de molécules d'air autrement les éléctrons s'entrechoquerait avec ces molécules ou seraient détournés puis s'écraseraient sur la paroi des tubes et ne pourraient plus servir aux expériences.
			
			Pour mettre sous vide, on utilise des pompes à spirales qui aspirent l'air puis l'évacuent vers l'extérieur. On passe de dix puissance vingt-quatre molécules par mètre cube à dix puissance dix-neuf molécules par mètre cube. On trouve des molécules également sur les parois. Elles peuvent se décoller et dégrader le vide rendant le tube inutilisable pour les expériences. Pour éviter cela,  ils chauffent les parois de cent degrés celsius à deux-cent degrés celsius pendant environ vingt-quatre heure. Après tout ces processus, il reste dix puissance onze molécules par mètres cubes. On utilise également des pompes ioniques ou statiques qui elles piègent les molécules. Même si le vide n'est pas encore parfait, il est suffisant pour mener à bien les expériences car la probabilité d'interaction d'un électron avec une molécule résiduelle est suffisament petite. 

			Si on pense qu'il y a une fuite quelque part dans un tube ou une pièce, on utilise un détecteur. On vaporise de l'hélium autour de la pièce et un détecteur nous informe par un graphique si l'hélium a pu entrer dans le tube. Si l'hélium est entré dans le tube il faut réparer la pièce abimée.  
		
		\section{Électronicien et Électrotechnicien}
			Les électroniciens et les électrotechniciens développent, mettent au point et maintiennent les aimants. Parfois ils travaillent pour d'autres pays.
		
		\section{Resources humaines}
			Les personnes travaillant dans le groupe ressources humaines s'occupent des salaires, aident les salariés si ils ont des problèmes, s'occupent de l'embaucheme des personnels et des stagiaires. Lorsqu'une personne est embauché, les personnes du groupe ressources humaines constituent un dossier électronique. 

		\section{Scientifiques}
			Les scientifiques travaillent dans les lignes de lumière pour faire des expériences. Les scientifiques utilisent la lumière émise par les électrons pour observer des échantillons qu'ils déposent dans la cabane d'expériences. Dans chaque ligne de lumière les scientifiques font des expériences différentes. J'ai visité plusieurs lignes de lumières comme MARS, LUCIA, DIFFABS et METROLOGIE.
			
			La ligne de lumière MARS a pour but d'étudier des échantillons de matière radioactives. Ils doivent se protéger pour ne pas être irradiés et tomber malades. Si un sachet transportant un élément radioactifs se perce, un instrument bippe pour prévenir les personnes du danger. Sur MARS, les scientifiques utilisent des rayons X. Ils vérifient que les échantillons ne sont pas trop dangereux pour l'homme pour savoir si les bâtiments contaminés par la radio activité sont encore utilisables ou non (par exemple d'anciennes centrales nucléaires).
			
			Sur LUCIA, les scientifiques travaillent la métérologie donc il travaille des minéraux.
			
			Les scientifiques de la ligne de lumières DIFFABS travaillent sur la cristallographie donc les cristaux. 
			
			Sur la ligne de lumière METROLOGIE, les scientifiques font de la lithographie. ILs font des gravures avec de la lumière.
			
			Pour devenir chercheur dans le domaine de la physique, il faut un Bac S, continuer des études universitaires jusqu'au niveau Master (BAC+5), puis passer un doctorat dans un laboratoire de recherche (BAC+8).



			\begin{figure}
  		 \centering
  		 \includegraphics[width=16cm]{Chapitre2/planlignedelumieressoleil}}
  		 \caption{Lignes de lumière}
	\end{figure}


	
