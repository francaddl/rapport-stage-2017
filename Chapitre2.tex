%%%%%%%%%%%%%%%%%%%%%%%%%%%%%%%%%%%%%%%%%%%%%%%%%%%%%%%%%%%%%%%%%%%%%%%%%%%%%%%%%%%%%%%%%%%%%
%%									Chapitre 2												%
%%%%%%%%%%%%%%%%%%%%%%%%%%%%%%%%%%%%%%%%%%%%%%%%%%%%%%%%%%%%%%%%%%%%%%%%%%%%%%%%%%%%%%%%%%%%%

\chapter{La formation}

	
	\minitoc
	

%%%%%%%%%%%%%%%%%%%%%%%%%%%%%%%%%%%%%%%%%%%%%%%%%%%%%%%%%%%%%%%%%%%%%%%%%%%%%%%%%%%%%%%%%%%%%



% Début du chapitre
			
	\section{Qualifications}

		\subsection
			\blindtext 
		\subsection
		\subsection{Coordinateurs expérience}
			Le coordinateur expérience assiste au bon fonctionnement de chaque ligne de lumière. Il doit pouvoir intrvenir dans tous les domainesde la physique. Il doit aussi connaître les gestes de premier secours et savoir utiliser les machines.
			Le coordinateur expérience est ausi là pour surveiller la machine très coûteuse pendant les week-end et la nuit quand les autres personnes ne travaillent pas.
			Au synchrotron SOLEIL il y a six coordinateurs expériences. Il se partage le travail en trois fois huit par journée, 	cela veut dire que chaque jour il y a trois coordinateur expériences qui travaillent; un travaille le matin, le deuxième travaille l'apès-midi et en fin de soirée et le troisième travaille pendant la nuit. Il s'échange le poste tous les jours (week-end et jours fériés). Pour que chacun puisse avoir des vacances ils sont six.
			Les formations pour ce poste sont scientifique: il faut un bac scientifique et avoir fait des études supplémentaires en physiques. Il faut parler anglais et connaître les termes spécifiques en anglais parce qu'au synchrotron SOLEIL, des chercheurs et scientifiques de tous les pays viennent y travailler.
			Le salaire est dénviron 3500 euros net par mois.
		\subsection{Mécaniciens}
			Lors de la construction de láccélérateur d'éléctrons, les mécaniciens ont installés les pièces et construient la machine. 
			Maintenant, il s'occupe de la maintenance dela machine et construisent des pièces avec différentes machines.
			Ils utilisent la fraiseuse et la tourneuse pour transformer des métaux en différentes pièces. Ils utilisent aussi une rectifieuse pour lisser les faces, une plieuse pour plier les différents métaux.Les mécaniciens soudent aussi.
			Le revnu est entre 25k et 30k euros brut annuels. La formation est un appretissage en école spécialisé, un CAP, un BEP, BP, Bac Pro.
		\subsection{Personnes s'occupant des achats}
			Il faut certaines personnes qui s'occupent des achats. Pour cela il y a une assisstante, trois acheteurs et deux personnes qui commandes. Les catégories d'achats sont les fournitures, les services, et les travaux.
		\subsection{Videurs}
			Certaines persones s'occcupent de vérifier si les tubes sont bien sous vide. C'est à dire que les tubes ne doivent pas contenir de molécules d'air autrement les éléctrons s'entrechoquerait avec les molécules ou les détournerait mais s'écraserait sur la paroi des tubes et ne pourrait plus servir aux expériences.
			Pour mettre sous vide, on utilise des à spirale qui aspire l'ar puis l'évacue vers l'extérieur. On passe de dix puissance vingt-quatre molécules à dix puissance dix-neuf molécules. Comme on trouve des molécules également sur les parrois donc si elle se décolle des parois, elles remplissent le vide donc le tube serait "impropre". Donc ils chauffent les parois de cent degrés celsius à deux-cent degrés celsius pendant nviron vingt-qatre heure. Après tout ces prosessus il reste dix puissance onze molécules. On utilise également des pompes ioniques ou statiques qui elles gardent les molécules.Même si il en reste beaucoup ce n'est pas grave parce que les molécules qui restent on plus d'espace pour se déplacer donc les éléctrons risquent moins de rencontrer des molécules d'air. Si on pense quíl y a une fuite quelque part dans un tube ou une pièce, on utilise un détecteur. On vaporise de l'hélium autour de la pièce et un détecteur nous informe par un graphique si l'hélium a pu entrer dans le tube. Si l'hélium est entré dans le tube il faut réparer la pièce abimée.  
		\subsection{Électronicien et Électrotechnicien}
			Les électroniciens et les électrotechniciens maintiennent les aimants, mettent au poimt les aimants et développent aussi les aimants. Parfois les électroniciens et électrotechniciens travaillent pour d'autres pays.

	\section{Évolution de carrière}
		\blindtext
		
