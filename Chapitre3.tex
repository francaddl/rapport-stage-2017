%%%%%%%%%%%%%%%%%%%%%%%%%%%%%%%%%%%%%%%%%%%%%%%%%%%%%%%%%%%%%%%%%%%%%%%%%%%%%%%%%%%%%%%%%%%%%
%%									Chapitre 1											%
%%%%%%%%%%%%%%%%%%%%%%%%%%%%%%%%%%%%%%%%%%%%%%%%%%%%%%%%%%%%%%%%%%%%%%%%%%%%%%%%%%%%%%%%%%%%%
\chapter{L'intérêt du stage}
	\minitoc
	

%%%%%%%%%%%%%%%%%%%%%%%%%%%%%%%%%%%%%%%%%%%%%%%%%%%%%%%%%%%%%%%%%%%%%%%%%%%%%%%%%%%%%%%%%%%%%



% Début du chapitre

\section{Les raisons de mon choix}
    Au synchrotron SOLEIL il y a plusieurs corps de métier dont la plupart sont liés à la physique. Cela m'a permis de découvrir différents métier dans un domaine qui m'intéresse.
	
\section{Le synchrotron Soleil}
    Au laboratoire SOLEIL, j'ai appris beaucoup de choses sur les atomes et la physique mais j'ai aussi pu découvrir d'autres métiers essentiels pour les chercheurs comme le métier de mécanicien. Si il n'y avait pas de mécaniciens à SOLEIL, les scientifiques ne pourraient pas faire d'expériences et il n'aurait pas été possible de construire le synchrotron SOLEIL ni de le maintenir en conditions d'exploitation jusqu'à aujourd'hui. 

\section{Les difficultés que j'ai rencontrées}
    Comme le laboratoire SOLEIL est un laboratoire de physique les mots employés ne sont pas toujours compréhensibles pour les stagiaires comme moi qui n'ont pa le niveau d'études suffisant. Les scientifiques qui utilisent ces mots tous les jours,  n'arrivait pas à nous les expliquer car pour eux ces termes sont simples et ils n'arrivent donc pas à les définir plus simplement. 
    
    Quand on est dans le bâtiment du synchrotron, il faut faire très attention que personne ne reste dans les cabanes d'expériences car les scientifiques utilisent dans certaines lignes de lumière comme MARS des rayonnement dangereux pour l'homme. Ils doivent donc être absolument certains que personne ne reste dans la cabane d'expériences.

%\bibliographystyle{francaissc}
%\bibliography{Chapitre3/Biblio}