%%%%%%%%%%%%%%%%%%%%%%%%%%%%%%%%%%%%%%%%%%%%%%%%%%%%%%%%%%%%%%%%%%%%%%%%%%%%%%%%%%%%%%%%%%%%%
%%									Chapitre 1											%
%%%%%%%%%%%%%%%%%%%%%%%%%%%%%%%%%%%%%%%%%%%%%%%%%%%%%%%%%%%%%%%%%%%%%%%%%%%%%%%%%%%%%%%%%%%%%
\chapter{L'intérêt du stage}
	\minitoc
	

%%%%%%%%%%%%%%%%%%%%%%%%%%%%%%%%%%%%%%%%%%%%%%%%%%%%%%%%%%%%%%%%%%%%%%%%%%%%%%%%%%%%%%%%%%%%%



% Début du chapitre

\section{Les raisons de mon choix}
    Au synchrotron SOLEIL il y a plusieurs corps de métier différents dont la plupart sont liés à la physique qui m'intéresse. Cela m'a permis de découvrir différents métier dans un domaine que j'apprécie.
	
\section{Le synchrotron Soleil}
    Au laboratoire SOLEIL j'ai appris beaucoup de choses sur les atomes et la physique mais j'ai aussi pu découvrir d'autres métiers essentiels pour les scientifiques comme le métier de mécaniciens. Si il n"y avait pas de mécaniciens à SOLEIL, les scienyifiques ne pourrait pas faire d'expériences et il n"y aurait eu aucun intérêt de construire le synchrotron SOLEIL.

\section{Les difficultés que j'ai rencontrées}
    Comme le laboratoire SOLEIL est un laboratoire de physique les mots employés ne sont pas toujours compréhensible et les scientifiques les utilisant n'arrivait pas à nous les expliquer car pour eux ces termes sont simples et ils n'arrivent donc pas à les définir. 
    
    Quand on est dans le bâtiment du synchrotron, il faut faire très attention que personne ne reste dans les cabanes d'expériences car les scientifiques utilisent dans certaines lignes de lumière comme MARS des rayonnement dangereux pour l'homme. Ils doivent donc vérifier que personnes ne restent dans la cabane d'expériences.

%\bibliographystyle{francaissc}
%\bibliography{Chapitre3/Biblio}